\documentclass{article}
\usepackage{graphicx} % Required for inserting images

\title{Ramanujan summations}
\author{QuasiStar}
\date{November 2023}
\usepackage{amssymb}

\begin{document}

\maketitle
\section{Formula}

$$\frac{f\left(0\right)}{2}+\sum_{n=1}^{\infty}f\left(n\right) \stackrel{R}{=} i\int_{0}^{\infty}\frac{f\left(it\right)-f\left(-it\right)}{e^{2\pi t}-1}dt$$

\section{Examples}
One example of this is the following, infinite divergent summation:
$$1^{2}+2^{2}+3^{2}+4^{2}+...$$
This can be represented as the function of n: $f(n)=n^{2}$, plugging this in we have:
$$0+\sum_{n=1}^{\infty}n^{2} \stackrel{R}{=} i\int_{0}^{\infty}\frac{-t^{2}-t^{2}}{e^{2\pi t}-1}dt$$
With this, we realise that this is a simple integration of 0, resulting in a constant $C$.
$$i\int_{0}^{\infty}0dt$$
Therefore, we have 
$$\sum_{n=1}^{\infty}n^{2} \stackrel{R}{=} 0$$


Another example of this is the infinite divergent summation:
$$1+2+3+4+5....$$
we can use the function $f(n)=n$:
$$0+\sum_{n=1}^{\infty}n \stackrel{R}{=}  i\int_{0}^{\infty}\frac{it-(-it)}{e^{2\pi t}-1}dt$$
$$=  i\int_{0}^{\infty}\frac{2it}{e^{2\pi t}-1}dt$$
since $i$ is considered a constant, we can multiply it out, therefore it is:
$$=  -2\int_{0}^{\infty}\frac{t}{e^{2\pi t}-1}dt$$
using u-substitution we have $u=2\pi t$ and $du=2\pi$. Taking this we can apply this to the function above:
$$-2\int_{0}^{\infty}\frac{\frac{u}{2\pi}}{e^{u}-1}\frac{du}{2\pi t} = -\frac{1}{\pi^{2}}\int_{0}^{\infty}\frac{u}{e^{u}-1}du$$
Seeing that this is the Bose integration problem with the solution of $\frac{\pi^{2}}{12}$ we can plug it in,

$$-\frac{1}{\pi^{2}}\cdot\frac{\pi^{2}}{12}=-\frac{1}{12}$$
And therefore 
$$1+2+3+4+...\stackrel{R}{=} -\frac{1}{12}$$

\end{document}
