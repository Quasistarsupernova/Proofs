\documentclass{article}
\usepackage{graphicx} % Required for inserting images

\title{Quantum Computing, probability}
\author{quasimommy-chan}
\date{November 2023}
\usepackage{amsmath}

\begin{document}

\maketitle

\section{Three polarizer paradox}

we can find the probability of a state by using the probability matrices listed below.

$$\begin{bmatrix}
    \begin{bmatrix}
        cos(\theta)\\
        -sin(\theta)
    \end{bmatrix}
    &,
    \begin{bmatrix}
        sin(\theta)\\
        cos(\theta)
    \end{bmatrix}\\
\end{bmatrix}$$

    Inputting $90\degree$ degrees, $0\degree$ degrees and $45\degree$ degrees we have:

$$\begin{bmatrix}
    \begin{bmatrix}
        0\\
        1
    \end{bmatrix}
    &
    \begin{bmatrix}
        1\\
        0
    \end{bmatrix}\\
\end{bmatrix},
\begin{bmatrix}
    \begin{bmatrix}
        1\\
        0
    \end{bmatrix}
    &
    \begin{bmatrix}
        0\\
        1
    \end{bmatrix}\\
\end{bmatrix},
\begin{bmatrix}
    \begin{bmatrix}
        \frac{\sqrt{2}}{2}\\
        -\frac{\sqrt{2}}{2}
    \end{bmatrix}
    &
    \begin{bmatrix}
        \frac{\sqrt{2}}{2}\\
        \frac{\sqrt{2}}{2}
    \end{bmatrix}\\
\end{bmatrix}$$

    Because of this, photons that pass through the first filter will be polarized vertically:  $\begin{bmatrix}
        1\\
        0
    \end{bmatrix}\\$, Otherwise known as $\ket{\uparrow}$. Since all photons pass through the first filter, there is still 100 percent of photons remaining. Passing through the $0\degree$ filter from the first filter results in 0 percent passing through. This is because: $$\begin{bmatrix}
        1\\
        0
    \end{bmatrix}\\ = \boxed{0}\begin{bmatrix}
        0\\
        1
    \end{bmatrix}\\ + 1\begin{bmatrix}
        1\\
        0
    \end{bmatrix}\\$$   Thus, because $0^{2}=0$, with a $90\degree$ and $0\degree$ polarizing lens, no photons will pass through. However, placing the second $45\degree$ filter between them, or $\ket{\nearrow}$ will change the probability. Looking below we see that $$\begin{bmatrix}
        1\\
        0
    \end{bmatrix}\\ = \boxed{\frac{1}{\sqrt{2}}}\begin{bmatrix}
        \frac{1}{\sqrt{2}}\\
        -\frac{1}{\sqrt{2}}
    \end{bmatrix}\\ + \frac{1}{\sqrt{2}}\begin{bmatrix}
        \frac{1}{\sqrt{2}}\\
        \frac{1}{\sqrt{2}}
    \end{bmatrix}\\$$
    Therefore there is a $(\frac{1}{\sqrt{2}})^{2} = \frac{1}{2}$ chance that a photon will pass through either lens. Now, we have to consider the final lens, $$\begin{bmatrix}
        \frac{1}{\sqrt{2}}\\
        -\frac{1}{\sqrt{2}}
    \end{bmatrix}\\ = \boxed{-\frac{1}{\sqrt{2}}}\begin{bmatrix}
        0\\
        1
    \end{bmatrix}\\+ \frac{1}{\sqrt{2}}\begin{bmatrix}
        1\\
        0
    \end{bmatrix}\\$$   Looking at this we realise that there is also a 50 percent chance of passing through the third lens from the second lens. We conclude, with the final note that only $$\frac{1}{2} \cdot \frac{1}{2} = \boxed{\frac{1}{4}}$$ of the photons will pass through all three lenses $\ket{\uparrow}, \ket{\rightarrow}, \ket{\nearrow}$. Proving the three-polarizer paradox.
\end{document}
